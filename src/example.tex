
\setRecipeColors{
    recipename = black,
    numeration = black
}

\setRecipeLengths{
    ingredientswidth = .55\textwidth
}

\setHeadlines
{% translation
    inghead = Zutaten,
    prephead = Zubereitung,
    hinthead = Tipp,
    continuationhead = Fortsetzung,
    continuationfoot = Fortsetzung auf nächster Seite,
    portionvalue = Personen,
}


\begin{recipe}
[%
    preparationtime = {\unit[1]{h}},
    bakingtime={\unit[1]{h}},
    bakingtemperature={\protect\bakingtemperature{
        fanoven=\unit[225]{\textcelcius},
        topbottomheat=\unit[230]{\textcelcius}}},
        %topheat=\unit[195]{\textcelcius},
        %gasstove=Level 2}},
    portion = {ergibt 3 Bleche},
    %calory={\unit[3]{kJ}},
    source = {Florentina Walser}
]
{Florentinas Zimtschnecken}

%    \graph
%    {% pictures
%        small=pic/glass,     % small picture
%        big=pic/ingredients  % big picture
%    }

    \introduction{%
        Florentinas berühmten Zimtschnecken gehören an einem kalten Wintertag zum Kaffee dazu. Dafür musst du aber nicht die nächste Bäckereiaufsuchen sondern kannst dir den Genuss selber backen.
    }

    \ingredients{%
        \textbf{Hefeteig:} \\
        \unit[100]{g} & Rohrzucker \\
        \unit[4]{dl} & Pflanzenmilch \\
        1 Würfel & Hefe \\
        \unit[120]{g} & Margarine oder \unit[100]{g} Rapsöl mit Buttergeschmak \\
        \unit[700]{g} & (Dinkel-) Halbweissmehl \\
        \unit[1-3]{EL} & Kardamon, gemahlen \\
        1 Prise & Salz \\
        \textbf{Füllung:} \\
        \unit[200]{g} & weiche Datteln, entstein \\
        \unit[200]{g} & zimmerwarme Margarine oder ca. \unit[150]{ml} Rapsöl mit Buttergeschmack \\
        \unit[1-2]{TL} & Salz \\
        viel & Zimt \\
        \unit[1]{TL} & Kardamom \\
        \unit[200]{g} & Rohrzucker \\
        \unit[100]{g} & gemahlene Mandeln \\
        \textbf{Glasur:} \\
        \unit[100]{ml} & Ahornsirup oder Dattelsirup \\
        & Hagelzucker zum Bestreuen
    }

    \preparation{%
        \begin{enumerate}
            \item [\textbf{Hefeteig:}]
            \item 1 TL Rohrzucker in 1 dl lauwarmer (zimmerwarmer!) Milch auflösen, Hefe reinbröckeln, gut mischen und mindestens 10 Min. stehen lassen.
            \item Margarine (oder Öl) und restliche Milch (4dl) in einem Pfännchen erwärmen, bis die Margarine geschmolzen ist.
            \item 800g Mehl, restlicher Zucker, Kardamom und Salz in einer grossen Schüssel mischen, Hefeansatz und Margarine-/Öl-Mischung zu den trockenen Zutaten geben, alles zu einem Teig verarbeiten, gut durchkneten (klebt er an den Fingern, etwas Mehl beigeben). Teig in eine bemehlte Schüssel geben und zugedeckt mindestens 1 Std. aufgehen lassen (je länger, desto besser!).
%            \item [\textbf{Backen:}]
%            \item Den Ofen auf 225 Grad vorheizen.
%            \item Den Hefeteig aus der Schüssel auf eine bemehlte Arbeitsfläche geben und nochmals gut durchkneten, falls nötig, mehr Mehl dazugeben. Den Teig in 2 Hälften teilen (sonst wird er unbewältigbar gross), folgende Schritte 2x ausführen.
%            \item Den Teig längs rechteckig und ca. 1cm dick auf genug Mehl auswallen. Er darf nicht an der Arbeitsfläche festkleben!
%            \item Die Dattelfüllung auf den Teig verteilen, mit einem grossen Löffel gemahlene Mandeln, Kardamom, Zucker und so viel Zimt wie erwünscht bestreuen. Den Teig von einer Seite in eine lange Rolle aufrollen, ca. 10-12 Stücke schneiden und in einer gefetteten Auflaufform nochmals zugedeckt 20 Minuten aufgehen lassen.
%            \item Anschliessend 10-15 Minuten oder bis die Schnecken goldbraun sind in der Ofenmitte backen.
%            \item Die fertig gebackenen Schnecken noch ofenheiss mit Ahornsirup bepinselnund mit Hagelzucker bestreuen, kurz abkühlen lassen. Am besten lauwarm mit einer Tasse schwarzem Kaffee geniessen. :)
        \end{enumerate}

        \begin{enumerate}
            \item [\textbf{Füllung:}]
            \item Datteln mit kochendem Wasser übergiessen und mindestens 30 Minuten einweichen lassen (am besten während der Teig aufgeht).
            \item Die Datteln mit dem Salz und der Margarine/Öl zu einer glatten Masse pürieren.
        \end{enumerate}

        \begin{enumerate}
            \item [\textbf{Backen:}]
            \item Den Ofen auf 225 Grad vorheizen.
            \item Den Hefeteig aus der Schüssel auf eine bemehlte Arbeitsfläche geben und nochmals gut durchkneten, falls nötig, mehr Mehl dazugeben. Den Teig in 2 Hälften teilen (sonst wird er unbewältigbar gross), folgende Schritte 2x ausführen.
            \item Den Teig längs rechteckig und ca. 1cm dick auf genug Mehl auswallen. Er darf nicht an der Arbeitsfläche festkleben!
            \item Die Dattelfüllung auf den Teig verteilen, mit einem grossen Löffel gemahlene Mandeln, Kardamom, Zucker und so viel Zimt wie erwünscht bestreuen. Den Teig von einer Seite in eine lange Rolle aufrollen, ca. 10-12 Stücke schneiden und in einer gefetteten Auflaufform nochmals zugedeckt 20 Minuten aufgehen lassen.
            \item Anschliessend 10-15 Minuten oder bis die Schnecken goldbraun sind in der Ofenmitte backen.
            \item Die fertig gebackenen Schnecken noch ofenheiss mit Ahornsirup bepinselnund mit Hagelzucker bestreuen, kurz abkühlen lassen. Am besten lauwarm mit einer Tasse schwarzem Kaffee geniessen. :)
        \end{enumerate}
    }

%    \suggestion[Headline]
%    {%
%        \blindtext
%    }
%
%    \suggestion{%
%        \blindtext
%    }

    \hint{%
        Für besonders schöne Schnecken: Die Dattelfüllung auf der unteren Hälfte des Teiges bis zur Mitte verteilen, mit Mandeln, Zucker und Gewürzen bestreuen, dann die obere Teighälfte nach unten klappen, sodass die Füllung bedeckt ist. Den nun eher schmalen, länglichen Teigstreifen in ca 2.5cm breite Streifen schneiden. Die Streifen von beiden Seiten gegengleich verdrehen, zu Schnecken aufrollen und das Ende unterstecken. Auf dem Blech zugedeckt nochmals 20 Minuten aufgehen lassen.
    }

\end{recipe}
